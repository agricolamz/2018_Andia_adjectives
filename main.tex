%!TEX TS-program = xelatex

\documentclass[a4paper,12pt]{article}
%\usepackage[english,russian]{babel} 
\usepackage{fontspec}
\setmainfont[Ligatures=TeX,
						Path=font/,
						BoldFont=brillb,
						ItalicFont=brilli,
						BoldItalicFont=brillbi]{brill}
\setsansfont[Ligatures=TeX,
						Path=font/,
						BoldFont=brillb,
						ItalicFont=brilli,
						BoldItalicFont=brillbi]{brill}

\newfontfamily\tgtermes{TeX Gyre Termes}
\makeatletter
  \begingroup
    \tgtermes
    \DeclareFontShape{\f@encoding}{\rmdefault}{m}{sc}{%
      <-> ssub * \f@family/m/sc}{}
    \DeclareFontShape{\f@encoding}{\rmdefault}{bx}{sc}{%
      <-> ssub * \f@family/bx/sc}{}
  \endgroup
\makeatother

\usepackage{subcaption}
\usepackage{hhline}
\usepackage{enumitem}
\setlist{nolistsep, leftmargin=5mm}
\usepackage{longtable}
\usepackage{multirow} 
\usepackage{multicol} 
\usepackage{setspace} 
%\onehalfspacing % Интерлиньяж 1.5
%\doublespacing % Интерлиньяж 2
\singlespacing % Интерлиньяж 1
\usepackage{geometry} % Простой способ задавать поля
\geometry{top=20mm}
\geometry{bottom=20mm}
\geometry{left=20mm}
\geometry{right=20mm}
\usepackage{hyperref}
\usepackage[usenames,dvipsnames,svgnames,table,rgb]{xcolor}
	\hypersetup{ % Гиперссылки
	colorlinks=true, % false: ссылки в рамках; true: цветные ссылки
	linkcolor=black, % внутренние ссылки
	citecolor=black, % на библиографию
	filecolor=black, % на файлы
	urlcolor=ForestGreen % на URL
	}

\usepackage{forest} % для рисования деревьев
\usepackage{vowel} % для рисования трапеций гласных
\usepackage[toc,page]{appendix}
\usepackage{config/leipzig}
\newleipzig{Abs}{abs}{absolutive case}
\newleipzig{Add}{ad}{Ad location}
\newleipzig{Addi}{add}{additive}
\newleipzig{Ads}{ad2}{Ad2 location}
\newleipzig{Aff}{aff}{affective case}
\newleipzig{An}{an}{animal class}
\newleipzig{Attr}{attr}{attributive particle}
\newleipzig{Aor}{aor}{aorist}
\newleipzig{Caus}{caus}{causative}
\newleipzig{Com}{com}{comitative}
\newleipzig{Dat}{dat}{dative case}
\newleipzig{Erg}{erg}{ergative case}
\newleipzig{Ess}{ess}{essive direction}
\newleipzig{Elat}{elat}{ellative direction}
\newleipzig{F}{f}{female class}
\newleipzig{Gen}{gen}{genitive case}
\newleipzig{H}{h}{human}
\newleipzig{Hab}{hab}{habitual}
\newleipzig{Imp}{imp}{imperative}
\newleipzig{Inst}{inst}{instrumental case}
\newleipzig{Indef}{indef}{indefinite}
\newleipzig{Inter}{inter}{Inter location}
\newleipzig{Jus}{jus}{jussive}
\newleipzig{Lat}{lat}{lative direction}
\newleipzig{M}{m}{masculine class}
\newleipzig{Nanf}{¬an1}{first non-animated class}
\newleipzig{Nans}{¬an2}{second non-animated class}
\newleipzig{Neg}{neg}{negation marker}
\newleipzig{Nh}{¬h}{non-human}
\newleipzig{Npst}{¬pst}{non-past forms}
\newleipzig{Obl}{obl}{oblique stem}
\newleipzig{Pf}{pf}{perfect tense}
\newleipzig{Pl}{pl}{plural}
\newleipzig{Pres}{pres}{present tense}
\newleipzig{Prog}{prog}{progressive}
\newleipzig{Pst}{pst}{past tense}
\newleipzig{Ptcp}{ptcp}{participle}
\newleipzig{Rfl}{refl}{reflexive pronoun}
\newleipzig{Super}{super}{super location}
\newleipzig{Wh}{wh}{special question marker}
\makeglossaries
\usepackage{natbib}
\bibpunct[: ]{[}{]}{;}{a}{}{,}
\usepackage{philex} % пакет для примеров
\renewcommand{\thesection}{\arabic{section}.}
\renewcommand{\thesubsection}{\arabic{section}.\arabic{subsection}}
	\setlength{\columnsep}{1.6cm}
	
\usepackage{footnote}
\makesavenoteenv{tabular}
\title{\Large Adjectives and pronouns in Zilo}
\author{G. Moroz}
\date{August 2018, last version}
\begin{document} 
\maketitle

\section{Adjectives}
Adjectives can be clearly distinguished from nouns or verbs since they are not lexically specified for gender, and cannot express any verbal categories without adding some derivational morphology (e. g. causative). The most of the Zilo adjectives have no class agreement marker. Class agreement marker appears initially~(\ref{initialy}) and finally (\ref{finally}):\footnote{Sometimes it appears internally, e. g. \textit{se}<\textsc{cl}>\textit{ɡulo} `no one', but in these cases it is easy to show that this is the result of the derivation.}

\ex. 	\ag. hirts'i woʃo\\
				tall boy\\
				\glt `Tall boy'.
		\bg. hirts'i joʃi\\
				tall girl\\
				\glt `Tall girl'.			 

\ex. \label{initialy}	
		\ag. w-etʃ'uχa woʃo\\
				{\M-big} boy\\
				\glt `Big boy'.
		\bg. j-etʃ'uχa joʃi\\
				{\F-big} girl\\
				\glt `Big girl'.

\ex. \label{finally}
		\ag. tsi-w woʃo\\
				{new-\M} boy\\
				\glt `New boy'.
		\bg. tsi-j joʃi\\
				{new-\F} girl\\
				\glt `New girl'.

\subsection{Derivation of adjectives}
The most easiest way to derive something attributive is to use participle form of verbs.

Genitive forms of nouns  are really productively used in the function fulfilled in other languages by relational adjectives, for example 

Reduplication

Zilo also have a very productive suffix -sːi that can change the whole clause to a headed or headless noun modifier and form a parallel with participles: 

\subsection{Adjectival inflection}
All adjectives have the same suffixal inflection. In combination with a noun they modify they take suffixes expressing plural number agreement with their head. When the noun that adjective is modifying is absent, adjectives are inflected for gender, number and case. The inflection is quite the same for all substantivised unites (e. g. numerals, participles etc.).  In absolutive the gender is not distinguished, but in ...



\section*{Glosses}
\small
\printglosses
\bibliographystyle{config/chicago}
\bibliography{config/bibliography}
\normalsize
\end{document}